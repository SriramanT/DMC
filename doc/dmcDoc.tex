\RequirePackage[l2tabu, orthodox]{nag}

\documentclass[letterpaper, 11 pt]{article}
%\documentclass[a5paper, 8 pt]{extreport}
%\usepackage[a5paper, top = 1.0 in, bottom = 0.9 in]{geometry}
%\usepackage[pass]{geometry}

%\usepackage[sc]{mathpazo}
%\linespread{1.05}

\usepackage{garamondx}
\usepackage[garamondx,cmbraces]{newtxmath}

\usepackage[T2A, T1]{fontenc}
\usepackage[utf8]{inputenc}
\usepackage{CJKutf8}
\usepackage{mathtools}
\usepackage{amsfonts}
\usepackage{amsmath}
\usepackage{amssymb}
\usepackage{braket}
\usepackage{ellipsis}
\usepackage{microtype}
\usepackage{graphicx}
\usepackage{placeins}
\usepackage{booktabs}
\usepackage{caption}
\usepackage{subcaption}
\usepackage[usenames,dvipsnames]{xcolor}
\usepackage{setspace}
\usepackage[sc]{titlesec}
\usepackage[doi = false, url = false, isbn = false,
            maxbibnames = 99, sorting = none, backend = bibtex,
            giveninits = true, style = numeric-comp]{biblatex}
\usepackage[colorlinks, breaklinks=true]{hyperref}
\usepackage[toc,page]{appendix}

%change link colours:
\hypersetup{ linktocpage, colorlinks=true, linkcolor=blue,
             citecolor=blue, filecolor=blue, urlcolor=blue }

%Bibliography stuff:
\addbibresource{dmcDoc.bib}

%Make titles a link:
\newbibmacro{string+doiurlisbn}[1]{%
   \iffieldundef{doi}{%
      \iffieldundef{url}{%
         \iffieldundef{isbn}{%
            \iffieldundef{issn}{%
               #1%
            }{%
               \href{http://books.google.com/books?vid=ISSN\thefield{issn}}{#1}%
            }%
         }{%
            \href{http://books.google.com/books?vid=ISBN\thefield{isbn}}{#1}%
         }%
      }{%
         \href{\thefield{url}}{#1}%
      }%
   }{%
      \href{http://dx.doi.org/\thefield{doi}}{#1}%
   }%
}

\DeclareFieldFormat{title}{\usebibmacro{string+doiurlisbn}{\mkbibemph{#1}}}
\DeclareFieldFormat[article,incollection]{title}%
   {\usebibmacro{string+doiurlisbn}{\mkbibquote{#1}}}


\title{PHYS 5000 Computational Assignment:\\The diffusion Monte Carlo method}
\date{Fall 2018}
\author{}

\begin{document}

\maketitle

\section*{Introduction}

Consider an $N$ particle quantum system with Hamiltonian
%
\begin{equation} \label{eq:ham}
   \hat{H} = \sum\limits_{j = 1}^N \frac{\hat{p}^2_j}{2m_j} + V(\hat{\mathbf{r}}_1, \dots
   \hat{\mathbf{r}_N}).
\end{equation}
%
A Hamiltonian of this form can be used to describe an atom or molecule with $N$ electrons. For such
systems it is often useful to know the ground-state energy $E_0$ and the ground-state wave function
$\phi_0$. For example, the ground-state energy can be used to determine ionization potentials, the
amount of energy needed to remove an electron from an atom/molecule. The ground state itself can be used
as the initial value in a variety of situations. Unfortunately it is often impossible to explicitly
determine these quantities.

Expanding a solution $\psi$ of the Schr\"odinger equation associated with the Eq.~\eqref{eq:ham} in terms
of energy eigenfunctions we have, in atomic units,
%
\begin{equation} \label{eq:expansion}
   \psi(\mathbf{r}_1,\dots,\mathbf{r}_N,t) = \sum_n c_n \phi_n(\mathbf{r}_1,\dots,\mathbf{r}_N)
   e^{-i E_n t},
\end{equation}
%
where
%
\begin{equation}
   \hat{H} \phi_n = E_n \phi_n,
\end{equation}
%
with
%
\begin{equation}
   E_0 < E_1 \leq E_2 \dots
\end{equation}

To facilitate the calculation of the ground-state energy and wave function we will make two changes.
First we perform a Wick rotation by making the substitution $\tau = it$. In which case the Schr\"odinger
equation becomes a diffusion equation
%
\begin{equation}
   \frac{\partial \psi}{\partial \tau} = \sum_j^N \frac{\Delta_j \psi}{2m_j} - V(x) \psi.
\end{equation}
%
Second, we shift the energy scale by replacing $V \rightarrow V - E_R$ for some $E_R$.

With both of these changes made Eq.~\eqref{eq:expansion} becomes
%
\begin{equation}
   \psi(\mathbf{r}_1,\dots,\mathbf{r}_N,\tau) = \sum_n c_n \phi_n(\mathbf{r}_1,\dots,\mathbf{r}_N)
   e^{-(E_n- E_r) \tau}.
\end{equation}
%
There are now three possible outcomes:

\begin{itemize}
      
   \item[(i)] If $E_R > E_0$ Then $\lim\limits_{\tau \rightarrow \infty} \psi(\tau) \rightarrow \infty$.

   \item[(ii)] If $E_R < E_0$ Then $\lim\limits_{\tau \rightarrow \infty} \psi(\tau) \rightarrow 0$.

   \item[(iii)] If $E_R = E_0$ Then $\lim\limits_{\tau \rightarrow \infty} \psi(\tau) \rightarrow
      c_0\phi_0$.

\end{itemize}
%
We then see that for the right choice of $E_R$ we can determine the ground state of the system by
propagating to infinite imaginary time. This is the basis of the diffusion Monte Carlo method~\cite{dmc} (\textsc{dmc}) where the imaginary time diffusion equation is solved self consistently to calculate
$E_0$ and $\phi_0$ simultaneously.

\section*{Assignment}

\begin{itemize}

   \item[(1)] Read Ref.~\cite{dmc} for a brief introduction to \textsc{dmc}.

   \item[(2)] Using the code provided\footnote{\href{https://github.com/m0baxter/DMC}
      {https://github.com/m0baxter/DMC}}, which produces the time evolution of $E_R(\tau)$ and the
      ground-state wave function, or by writing your own, complete the following tasks:

      \begin{itemize}

         \item[(i)] Determine the ground-state energy of the helium atom. Estimate the error,
            $\Delta E_0$.

         \item[(ii)] Convince your self that determining the 6-dimensional wave function $\phi_0$
            is a non-trivial task\footnote{WARNING: Running the provided code using a large grid
            may quickly eat up all available \textsc{ram}.}.

         \item[(iii)] Generally working with the full Hamiltonian for an atomic/molecular system is
            simply too expensive. In practice one often works instead with a simplified \emph{soft-core}
            potential
            %
            \begin{equation}
               V(x_1, x_2) = -\frac{2}{\sqrt{{x_1}^2 + \epsilon_{en}}}
                             -\frac{2}{\sqrt{{x_2}^2 + \epsilon_{en}}}
                             -\frac{1}{\sqrt{(x_2 - x_1)^2 + \epsilon_{ee}}}
            \end{equation}
            %
            The softening parameters $\epsilon_{en}$ and $\epsilon_{ee}$ are then chosen to reproduce
            some properties of the full system.

            A popular choice are the ionization potentials. The $k$-th ionization potential for an
            atom/molecule is defined as the amount of energy required to remove $k$ electrons. This can
            be calculated by taking the difference between the ground-state energies of the original
            system and the ionized system. For helium we have
            %
            \begin{align}
               U_I^{(1)}(\mathrm{He}) = & E_0^{\mathrm{He}^+} - E_0^{\mathrm{He}}, \\
               U_I^{(2)}(\mathrm{He}) = & E_0^{\mathrm{He}^{2+}} - E_0^{\mathrm{He}^+} = 2.
            \end{align}

            Find values for $\epsilon_{en}$ and $\epsilon_{ee}$ such that the first and second
            ionization potential for the soft-core helium atom agree with those of the real helium atom.
            As a hint, by considering the goal of softening parameters one can argue that
            $\epsilon_{en}, \epsilon_{ee} \in (0,1)$. Provide a plot of the wave function
            $\phi_0(x_1, x_2)$.

         \item[(iv)] Write a report in the style of a journal article\footnote{If required a \LaTeX\
            template may be found \href{https://journals.aps.org/revtex}{here}.}. This should contain an
            introduction, a brief summary of the theory (a  description of \textsc{dmc}), a section
            summarizing results including a discussion of ways to minimize error, and a conclusion.

            Include with your submission all code written in a clear easily perusable format.

      \end{itemize}

\end{itemize}

\printbibliography[title=References]

\end{document}
